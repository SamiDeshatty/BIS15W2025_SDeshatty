% Options for packages loaded elsewhere
\PassOptionsToPackage{unicode}{hyperref}
\PassOptionsToPackage{hyphens}{url}
%
\documentclass[
]{article}
\usepackage{amsmath,amssymb}
\usepackage{iftex}
\ifPDFTeX
  \usepackage[T1]{fontenc}
  \usepackage[utf8]{inputenc}
  \usepackage{textcomp} % provide euro and other symbols
\else % if luatex or xetex
  \usepackage{unicode-math} % this also loads fontspec
  \defaultfontfeatures{Scale=MatchLowercase}
  \defaultfontfeatures[\rmfamily]{Ligatures=TeX,Scale=1}
\fi
\usepackage{lmodern}
\ifPDFTeX\else
  % xetex/luatex font selection
\fi
% Use upquote if available, for straight quotes in verbatim environments
\IfFileExists{upquote.sty}{\usepackage{upquote}}{}
\IfFileExists{microtype.sty}{% use microtype if available
  \usepackage[]{microtype}
  \UseMicrotypeSet[protrusion]{basicmath} % disable protrusion for tt fonts
}{}
\makeatletter
\@ifundefined{KOMAClassName}{% if non-KOMA class
  \IfFileExists{parskip.sty}{%
    \usepackage{parskip}
  }{% else
    \setlength{\parindent}{0pt}
    \setlength{\parskip}{6pt plus 2pt minus 1pt}}
}{% if KOMA class
  \KOMAoptions{parskip=half}}
\makeatother
\usepackage{xcolor}
\usepackage[margin=1in]{geometry}
\usepackage{color}
\usepackage{fancyvrb}
\newcommand{\VerbBar}{|}
\newcommand{\VERB}{\Verb[commandchars=\\\{\}]}
\DefineVerbatimEnvironment{Highlighting}{Verbatim}{commandchars=\\\{\}}
% Add ',fontsize=\small' for more characters per line
\usepackage{framed}
\definecolor{shadecolor}{RGB}{248,248,248}
\newenvironment{Shaded}{\begin{snugshade}}{\end{snugshade}}
\newcommand{\AlertTok}[1]{\textcolor[rgb]{0.94,0.16,0.16}{#1}}
\newcommand{\AnnotationTok}[1]{\textcolor[rgb]{0.56,0.35,0.01}{\textbf{\textit{#1}}}}
\newcommand{\AttributeTok}[1]{\textcolor[rgb]{0.13,0.29,0.53}{#1}}
\newcommand{\BaseNTok}[1]{\textcolor[rgb]{0.00,0.00,0.81}{#1}}
\newcommand{\BuiltInTok}[1]{#1}
\newcommand{\CharTok}[1]{\textcolor[rgb]{0.31,0.60,0.02}{#1}}
\newcommand{\CommentTok}[1]{\textcolor[rgb]{0.56,0.35,0.01}{\textit{#1}}}
\newcommand{\CommentVarTok}[1]{\textcolor[rgb]{0.56,0.35,0.01}{\textbf{\textit{#1}}}}
\newcommand{\ConstantTok}[1]{\textcolor[rgb]{0.56,0.35,0.01}{#1}}
\newcommand{\ControlFlowTok}[1]{\textcolor[rgb]{0.13,0.29,0.53}{\textbf{#1}}}
\newcommand{\DataTypeTok}[1]{\textcolor[rgb]{0.13,0.29,0.53}{#1}}
\newcommand{\DecValTok}[1]{\textcolor[rgb]{0.00,0.00,0.81}{#1}}
\newcommand{\DocumentationTok}[1]{\textcolor[rgb]{0.56,0.35,0.01}{\textbf{\textit{#1}}}}
\newcommand{\ErrorTok}[1]{\textcolor[rgb]{0.64,0.00,0.00}{\textbf{#1}}}
\newcommand{\ExtensionTok}[1]{#1}
\newcommand{\FloatTok}[1]{\textcolor[rgb]{0.00,0.00,0.81}{#1}}
\newcommand{\FunctionTok}[1]{\textcolor[rgb]{0.13,0.29,0.53}{\textbf{#1}}}
\newcommand{\ImportTok}[1]{#1}
\newcommand{\InformationTok}[1]{\textcolor[rgb]{0.56,0.35,0.01}{\textbf{\textit{#1}}}}
\newcommand{\KeywordTok}[1]{\textcolor[rgb]{0.13,0.29,0.53}{\textbf{#1}}}
\newcommand{\NormalTok}[1]{#1}
\newcommand{\OperatorTok}[1]{\textcolor[rgb]{0.81,0.36,0.00}{\textbf{#1}}}
\newcommand{\OtherTok}[1]{\textcolor[rgb]{0.56,0.35,0.01}{#1}}
\newcommand{\PreprocessorTok}[1]{\textcolor[rgb]{0.56,0.35,0.01}{\textit{#1}}}
\newcommand{\RegionMarkerTok}[1]{#1}
\newcommand{\SpecialCharTok}[1]{\textcolor[rgb]{0.81,0.36,0.00}{\textbf{#1}}}
\newcommand{\SpecialStringTok}[1]{\textcolor[rgb]{0.31,0.60,0.02}{#1}}
\newcommand{\StringTok}[1]{\textcolor[rgb]{0.31,0.60,0.02}{#1}}
\newcommand{\VariableTok}[1]{\textcolor[rgb]{0.00,0.00,0.00}{#1}}
\newcommand{\VerbatimStringTok}[1]{\textcolor[rgb]{0.31,0.60,0.02}{#1}}
\newcommand{\WarningTok}[1]{\textcolor[rgb]{0.56,0.35,0.01}{\textbf{\textit{#1}}}}
\usepackage{graphicx}
\makeatletter
\def\maxwidth{\ifdim\Gin@nat@width>\linewidth\linewidth\else\Gin@nat@width\fi}
\def\maxheight{\ifdim\Gin@nat@height>\textheight\textheight\else\Gin@nat@height\fi}
\makeatother
% Scale images if necessary, so that they will not overflow the page
% margins by default, and it is still possible to overwrite the defaults
% using explicit options in \includegraphics[width, height, ...]{}
\setkeys{Gin}{width=\maxwidth,height=\maxheight,keepaspectratio}
% Set default figure placement to htbp
\makeatletter
\def\fps@figure{htbp}
\makeatother
\setlength{\emergencystretch}{3em} % prevent overfull lines
\providecommand{\tightlist}{%
  \setlength{\itemsep}{0pt}\setlength{\parskip}{0pt}}
\setcounter{secnumdepth}{-\maxdimen} % remove section numbering
\ifLuaTeX
  \usepackage{selnolig}  % disable illegal ligatures
\fi
\usepackage{bookmark}
\IfFileExists{xurl.sty}{\usepackage{xurl}}{} % add URL line breaks if available
\urlstyle{same}
\hypersetup{
  pdftitle={Midterm 2 W25},
  pdfauthor={Sameeksha Deshatty},
  hidelinks,
  pdfcreator={LaTeX via pandoc}}

\title{Midterm 2 W25}
\author{Sameeksha Deshatty}
\date{2025-03-04}

\begin{document}
\maketitle

\subsection{Instructions}\label{instructions}

Before starting the exam, you need to follow the instructions in
\texttt{02\_midterm2\_cleaning.Rmd} to clean the data. Once you have
cleaned the data and produced the \texttt{heart.csv} file, you can start
the exam.

Answer the following questions and complete the exercises in RMarkdown.
Please embed all of your code and push your final work to your
repository. Your code must be organized, clean, and run free from
errors. Remember, you must remove the \texttt{\#} for any included code
chunks to run. Be sure to add your name to the author header above.

Your code must knit in order to be considered. If you are stuck and
cannot answer a question, then comment out your code and knit the
document. You may use your notes, labs, and homework to help you
complete this exam. Do not use any other resources- including AI
assistance or other students' work.

Don't forget to answer any questions that are asked in the prompt! Each
question must be coded; it cannot be answered by a sort in a spreadsheet
or a written response.

All plots should be clean, with appropriate labels, and consistent
aesthetics. Poorly labeled or messy plots will receive a penalty. Your
plots should be in color and look professional!

Be sure to push your completed midterm to your repository and upload the
document to Gradescope. This exam is worth 30 points.

\subsection{Load the libraries}\label{load-the-libraries}

You may not use all of these, but they are here for convenience.

\begin{Shaded}
\begin{Highlighting}[]
\FunctionTok{library}\NormalTok{(}\StringTok{"tidyverse"}\NormalTok{)}
\FunctionTok{library}\NormalTok{(}\StringTok{"janitor"}\NormalTok{)}
\FunctionTok{library}\NormalTok{(}\StringTok{"ggthemes"}\NormalTok{)}
\FunctionTok{library}\NormalTok{(}\StringTok{"RColorBrewer"}\NormalTok{)}
\FunctionTok{library}\NormalTok{(}\StringTok{"paletteer"}\NormalTok{)}
\end{Highlighting}
\end{Shaded}

\subsection{Load the data}\label{load-the-data}

These data are a modified version of the Statlog (Heart) database on
heart disease from the
\href{https://archive.ics.uci.edu/dataset/145/statlog+heart}{UCI Machine
Learning Repository}. The data are also available on
\href{https://www.kaggle.com/datasets/ritwikb3/heart-disease-statlog/data}{Kaggle}.

You will need the descriptions of the variables to answer the questions.
Please reference \texttt{03\_midterm2\_descriptions.Rmd} for details.

Run the following to load the data.

\begin{Shaded}
\begin{Highlighting}[]
\NormalTok{heart }\OtherTok{\textless{}{-}} \FunctionTok{read\_csv}\NormalTok{(}\StringTok{"data/heart.csv"}\NormalTok{)}
\end{Highlighting}
\end{Shaded}

\subsection{Questions}\label{questions}

Problem 1. (1 point) Use the function of your choice to provide a data
summary.

\begin{Shaded}
\begin{Highlighting}[]
\FunctionTok{head}\NormalTok{(heart)}
\end{Highlighting}
\end{Shaded}

\begin{verbatim}
## # A tibble: 6 x 14
##     age gender cp       trestbps  chol fbs   restecg thalach exang oldpeak slope
##   <dbl> <chr>  <chr>       <dbl> <dbl> <lgl> <chr>     <dbl> <chr>   <dbl> <chr>
## 1    70 male   asympto~      130   322 FALSE left_v~     109 no        2.4 flat 
## 2    67 female non_ang~      115   564 FALSE left_v~     160 no        1.6 flat 
## 3    57 male   atypica~      124   261 FALSE normal      141 no        0.3 upsl~
## 4    64 male   asympto~      128   263 FALSE normal      105 yes       0.2 flat 
## 5    74 female atypica~      120   269 FALSE left_v~     121 yes       0.2 upsl~
## 6    65 male   asympto~      120   177 FALSE normal      140 no        0.4 upsl~
## # i 3 more variables: ca <dbl>, thal <chr>, target <chr>
\end{verbatim}

\begin{Shaded}
\begin{Highlighting}[]
\FunctionTok{glimpse}\NormalTok{(heart)}
\end{Highlighting}
\end{Shaded}

\begin{verbatim}
## Rows: 270
## Columns: 14
## $ age      <dbl> 70, 67, 57, 64, 74, 65, 56, 59, 60, 63, 59, 53, 44, 61, 57, 7~
## $ gender   <chr> "male", "female", "male", "male", "female", "male", "male", "~
## $ cp       <chr> "asymptomatic", "non_anginal_pain", "atypical_angina", "asymp~
## $ trestbps <dbl> 130, 115, 124, 128, 120, 120, 130, 110, 140, 150, 135, 142, 1~
## $ chol     <dbl> 322, 564, 261, 263, 269, 177, 256, 239, 293, 407, 234, 226, 2~
## $ fbs      <lgl> FALSE, FALSE, FALSE, FALSE, FALSE, FALSE, TRUE, FALSE, FALSE,~
## $ restecg  <chr> "left_ventricular_hypertrophy", "left_ventricular_hypertrophy~
## $ thalach  <dbl> 109, 160, 141, 105, 121, 140, 142, 142, 170, 154, 161, 111, 1~
## $ exang    <chr> "no", "no", "no", "yes", "yes", "no", "yes", "yes", "no", "no~
## $ oldpeak  <dbl> 2.4, 1.6, 0.3, 0.2, 0.2, 0.4, 0.6, 1.2, 1.2, 4.0, 0.5, 0.0, 0~
## $ slope    <chr> "flat", "flat", "upsloping", "flat", "upsloping", "upsloping"~
## $ ca       <dbl> 3, 0, 0, 1, 1, 0, 1, 1, 2, 3, 0, 0, 0, 2, 1, 0, 2, 0, 0, 0, 2~
## $ thal     <chr> "normal", "reversable_defect", "reversable_defect", "reversab~
## $ target   <chr> "disease", "no_disease", "disease", "no_disease", "no_disease~
\end{verbatim}

\begin{Shaded}
\begin{Highlighting}[]
\FunctionTok{summary}\NormalTok{(heart)}
\end{Highlighting}
\end{Shaded}

\begin{verbatim}
##       age           gender               cp               trestbps    
##  Min.   :29.00   Length:270         Length:270         Min.   : 94.0  
##  1st Qu.:48.00   Class :character   Class :character   1st Qu.:120.0  
##  Median :55.00   Mode  :character   Mode  :character   Median :130.0  
##  Mean   :54.43                                         Mean   :131.3  
##  3rd Qu.:61.00                                         3rd Qu.:140.0  
##  Max.   :77.00                                         Max.   :200.0  
##       chol          fbs            restecg             thalach     
##  Min.   :126.0   Mode :logical   Length:270         Min.   : 71.0  
##  1st Qu.:213.0   FALSE:230       Class :character   1st Qu.:133.0  
##  Median :245.0   TRUE :40        Mode  :character   Median :153.5  
##  Mean   :249.7                                      Mean   :149.7  
##  3rd Qu.:280.0                                      3rd Qu.:166.0  
##  Max.   :564.0                                      Max.   :202.0  
##     exang              oldpeak        slope                 ca        
##  Length:270         Min.   :0.00   Length:270         Min.   :0.0000  
##  Class :character   1st Qu.:0.00   Class :character   1st Qu.:0.0000  
##  Mode  :character   Median :0.80   Mode  :character   Median :0.0000  
##                     Mean   :1.05                      Mean   :0.6704  
##                     3rd Qu.:1.60                      3rd Qu.:1.0000  
##                     Max.   :6.20                      Max.   :3.0000  
##      thal              target         
##  Length:270         Length:270        
##  Class :character   Class :character  
##  Mode  :character   Mode  :character  
##                                       
##                                       
## 
\end{verbatim}

Problem 2. (1 point) Let's explore the demographics of participants
included in the study. What is the number of males and females? Show
this as a table.

\begin{Shaded}
\begin{Highlighting}[]
\NormalTok{demographics }\OtherTok{\textless{}{-}}\NormalTok{ heart }\SpecialCharTok{\%\textgreater{}\%}
  \FunctionTok{count}\NormalTok{(gender, }\AttributeTok{sort =} \ConstantTok{TRUE}\NormalTok{)}

\NormalTok{demographics}
\end{Highlighting}
\end{Shaded}

\begin{verbatim}
## # A tibble: 2 x 2
##   gender     n
##   <chr>  <int>
## 1 male     183
## 2 female    87
\end{verbatim}

There are 183 male participants and 87 female participants in this
study.

Problem 3. (2 points) What is the average age of participants by gender?
Show this as a table.

\begin{Shaded}
\begin{Highlighting}[]
\NormalTok{avg\_age }\OtherTok{\textless{}{-}}\NormalTok{ heart }\SpecialCharTok{\%\textgreater{}\%}
  \FunctionTok{group\_by}\NormalTok{(gender) }\SpecialCharTok{\%\textgreater{}\%}
  \FunctionTok{summarize}\NormalTok{(}\AttributeTok{mean\_age=}\FunctionTok{mean}\NormalTok{(age, }\AttributeTok{na.rm=}\NormalTok{T))}

\NormalTok{avg\_age}
\end{Highlighting}
\end{Shaded}

\begin{verbatim}
## # A tibble: 2 x 2
##   gender mean_age
##   <chr>     <dbl>
## 1 female     55.7
## 2 male       53.8
\end{verbatim}

The average age of female participats is 55.68 and the average age of
male participants is 53.84

Problem 4. (1 point) Among males and females, how many have/do not have
heart disease? Show this as a table, grouped by gender.

\begin{Shaded}
\begin{Highlighting}[]
\NormalTok{disease\_target\_gender }\OtherTok{\textless{}{-}}\NormalTok{ heart }\SpecialCharTok{\%\textgreater{}\%}
  \FunctionTok{group\_by}\NormalTok{(gender) }\SpecialCharTok{\%\textgreater{}\%}
  \FunctionTok{count}\NormalTok{(target, }\AttributeTok{sort=} \ConstantTok{TRUE}\NormalTok{)}

\NormalTok{disease\_target\_gender}
\end{Highlighting}
\end{Shaded}

\begin{verbatim}
## # A tibble: 4 x 3
## # Groups:   gender [2]
##   gender target         n
##   <chr>  <chr>      <int>
## 1 male   disease      100
## 2 male   no_disease    83
## 3 female no_disease    67
## 4 female disease       20
\end{verbatim}

Among all male participants: 100 of the have heart disease, while 83 of
them do not. Among all female participants: 20 of them have heart
disease, while 20 of them do not.

Problem 5. (4 points) What is the percentage of males and females with
heart disease? Show this as a table, grouped by gender.

\begin{Shaded}
\begin{Highlighting}[]
\CommentTok{\#percent\_target\_gender \textless{}{-} heart \%\textgreater{}\%}
  \CommentTok{\#group\_by(gender) \%\textgreater{}\%}
  \CommentTok{\#count(target, sort= TRUE) \%\textgreater{}\%}
  \CommentTok{\#filter(target == "disease") \%\textgreater{}\%}
  \CommentTok{\#mutate(per=(n/demographics))}

\CommentTok{\#percent\_target\_gender}
\end{Highlighting}
\end{Shaded}

Problem 6. (3 points) Make a plot that shows the results of your
analysis from problem 5. If you couldn't get the percentages to work,
then make a plot that shows the number of participants with and without
heart disease by gender.

\begin{Shaded}
\begin{Highlighting}[]
\CommentTok{\#disease\_target\_gender \textless{}{-} heart \%\textgreater{}\%}
  \CommentTok{\#group\_by(gender) \%\textgreater{}\%}
  \CommentTok{\#count(target, sort= TRUE)\%\textgreater{}\%}
  \CommentTok{\#ggplot(data=heart, aes(x=gender, y=target))+}
  \CommentTok{\#geom\_col()}

\CommentTok{\#disease\_target\_gender}
\end{Highlighting}
\end{Shaded}

Problem 7. (3 points) Is there a relationship between age and
cholesterol levels? Make a plot that shows this relationship separated
by gender (hint: use faceting or make two plots). Be sure to add a line
of best fit (linear regression line).

\begin{Shaded}
\begin{Highlighting}[]
\NormalTok{age\_chol\_male }\OtherTok{\textless{}{-}}\NormalTok{ heart}\SpecialCharTok{\%\textgreater{}\%}
  \FunctionTok{group\_by}\NormalTok{(gender)}\SpecialCharTok{\%\textgreater{}\%}
  \FunctionTok{filter}\NormalTok{(gender }\SpecialCharTok{==} \StringTok{"male"}\NormalTok{)}
\FunctionTok{ggplot}\NormalTok{(}\AttributeTok{data=}\NormalTok{age\_chol\_male, }\AttributeTok{mapping=}\FunctionTok{aes}\NormalTok{(}\AttributeTok{x=}\NormalTok{age, }\AttributeTok{y=}\NormalTok{chol))}\SpecialCharTok{+}
  \FunctionTok{geom\_point}\NormalTok{()}\SpecialCharTok{+}
  \FunctionTok{geom\_smooth}\NormalTok{(}\AttributeTok{method=}\NormalTok{lm, }\AttributeTok{se=}\NormalTok{F, }\AttributeTok{na.rm=}\NormalTok{F)}\SpecialCharTok{+}
  \FunctionTok{labs}\NormalTok{(}\AttributeTok{title =} \StringTok{"Cholesterol Levels by Age {-} Male"}\NormalTok{, }\AttributeTok{x =} \StringTok{"Age"}\NormalTok{, }\AttributeTok{y =} \StringTok{"Cholesterol"}\NormalTok{)}
\end{Highlighting}
\end{Shaded}

\begin{verbatim}
## `geom_smooth()` using formula = 'y ~ x'
\end{verbatim}

\includegraphics{01_midterm2_exam_files/figure-latex/unnamed-chunk-9-1.pdf}

\begin{Shaded}
\begin{Highlighting}[]
\NormalTok{age\_chol\_female }\OtherTok{\textless{}{-}}\NormalTok{ heart}\SpecialCharTok{\%\textgreater{}\%}
  \FunctionTok{group\_by}\NormalTok{(gender)}\SpecialCharTok{\%\textgreater{}\%}
  \FunctionTok{filter}\NormalTok{(gender }\SpecialCharTok{==} \StringTok{"female"}\NormalTok{)}
\FunctionTok{ggplot}\NormalTok{(}\AttributeTok{data=}\NormalTok{age\_chol\_female, }\AttributeTok{mapping=}\FunctionTok{aes}\NormalTok{(}\AttributeTok{x=}\NormalTok{age, }\AttributeTok{y=}\NormalTok{chol))}\SpecialCharTok{+}
  \FunctionTok{geom\_point}\NormalTok{()}\SpecialCharTok{+}
  \FunctionTok{geom\_smooth}\NormalTok{(}\AttributeTok{method=}\NormalTok{lm, }\AttributeTok{se=}\NormalTok{F, }\AttributeTok{na.rm=}\NormalTok{F)}\SpecialCharTok{+}
  \FunctionTok{labs}\NormalTok{(}\AttributeTok{title =} \StringTok{"Cholesterol Levels by Age {-} Female"}\NormalTok{, }\AttributeTok{x =} \StringTok{"Age"}\NormalTok{, }\AttributeTok{y =} \StringTok{"Cholesterol"}\NormalTok{)}
\end{Highlighting}
\end{Shaded}

\begin{verbatim}
## `geom_smooth()` using formula = 'y ~ x'
\end{verbatim}

\includegraphics{01_midterm2_exam_files/figure-latex/unnamed-chunk-9-2.pdf}

Problem 8. (3 points) What is the range of resting blood pressure for
participants by type of chest pain? Make a plot that shows this
information.

\begin{Shaded}
\begin{Highlighting}[]
\NormalTok{bps }\OtherTok{\textless{}{-}}\NormalTok{ heart}\SpecialCharTok{\%\textgreater{}\%}
  \FunctionTok{group\_by}\NormalTok{(cp)}\SpecialCharTok{\%\textgreater{}\%}
  \FunctionTok{count}\NormalTok{(trestbps, }\AttributeTok{sort =} \ConstantTok{TRUE}\NormalTok{)}\SpecialCharTok{\%\textgreater{}\%}
  \FunctionTok{ggplot}\NormalTok{(}\AttributeTok{data=}\NormalTok{ heart, }\AttributeTok{mapping=}\FunctionTok{aes}\NormalTok{(}\AttributeTok{x=}\NormalTok{cp, }\AttributeTok{y=}\NormalTok{trestbps))}\SpecialCharTok{+}
  \FunctionTok{geom\_boxplot}\NormalTok{()}\SpecialCharTok{+}
  \FunctionTok{labs}\NormalTok{(}\AttributeTok{title =} \StringTok{"Resting Blood Pressure Range by Type of Chest Pain"}\NormalTok{, }\AttributeTok{x =} \StringTok{"Chest Pain Type"}\NormalTok{, }\AttributeTok{y =} \StringTok{"Resting Blood Pressure Range"}\NormalTok{)}

\NormalTok{bps}
\end{Highlighting}
\end{Shaded}

\includegraphics{01_midterm2_exam_files/figure-latex/unnamed-chunk-10-1.pdf}

Problem 9. (4 points) What is the distribution of maximum heart rate
achieved, separated by gender and whether or not the patient has heart
disease? Make a plot that shows this information- you must use faceting.

\begin{Shaded}
\begin{Highlighting}[]
\NormalTok{heart }\SpecialCharTok{\%\textgreater{}\%}
  \FunctionTok{group\_by}\NormalTok{(gender, target) }\SpecialCharTok{\%\textgreater{}\%}
  \FunctionTok{count}\NormalTok{(thalach, }\AttributeTok{sort =} \ConstantTok{TRUE}\NormalTok{)}\SpecialCharTok{\%\textgreater{}\%}
    \FunctionTok{ggplot}\NormalTok{(}\FunctionTok{aes}\NormalTok{(}\AttributeTok{x=}\NormalTok{gender, }\AttributeTok{y=}\NormalTok{thalach))}\SpecialCharTok{+}
    \FunctionTok{geom\_col}\NormalTok{()}
\end{Highlighting}
\end{Shaded}

\includegraphics{01_midterm2_exam_files/figure-latex/unnamed-chunk-11-1.pdf}

Problem 10. (4 points) What is the range of ST depression (oldpeak) by
the number of major vessels colored by fluoroscopy (ca)? Make a plot
that shows this relationship. (hint: should ca be a factor or numeric
variable?)

\begin{Shaded}
\begin{Highlighting}[]
\NormalTok{range\_dep }\OtherTok{\textless{}{-}}\NormalTok{ heart}\SpecialCharTok{\%\textgreater{}\%}
  \FunctionTok{group\_by}\NormalTok{(ca)}\SpecialCharTok{\%\textgreater{}\%}
  \FunctionTok{count}\NormalTok{(oldpeak, }\AttributeTok{sort =} \ConstantTok{TRUE}\NormalTok{)}\SpecialCharTok{\%\textgreater{}\%}
  \FunctionTok{ggplot}\NormalTok{(}\AttributeTok{data=}\NormalTok{ heart, }\AttributeTok{mapping=}\FunctionTok{aes}\NormalTok{(}\AttributeTok{x=}\NormalTok{oldpeak, }\AttributeTok{y=}\NormalTok{ca))}\SpecialCharTok{+}
  \FunctionTok{geom\_col}\NormalTok{()}\SpecialCharTok{+}
  \FunctionTok{labs}\NormalTok{(}\AttributeTok{title =} \StringTok{"Range of ST Depression"}\NormalTok{, }\AttributeTok{x =} \StringTok{"Fluroscopy"}\NormalTok{, }\AttributeTok{y =} \StringTok{"St Depression"}\NormalTok{)}

\NormalTok{range\_dep}
\end{Highlighting}
\end{Shaded}

\includegraphics{01_midterm2_exam_files/figure-latex/unnamed-chunk-12-1.pdf}

Problem 11. (4 points) Is there an age group where we see increased
prevalence of heart disease? Make a plot that shows the number of
participants with and without heart disease by age group.

\subsection{Submit the Midterm}\label{submit-the-midterm}

\begin{enumerate}
\def\labelenumi{\arabic{enumi}.}
\tightlist
\item
  Save your work and knit the .rmd file.\\
\item
  Open the .html file and ``print'' it to a .pdf file in Google Chrome
  (not Safari).\\
\item
  Go to the class Canvas page and open Gradescope.\\
\item
  Submit your .pdf file to the midterm assignment- be sure to assign the
  pages to the correct questions.\\
\item
  Commit and push your work to your repository.
\end{enumerate}

\end{document}
